%
%   Timed Mobile Systems
%   Andrew Hughes and Mike Stannett
%   Basic Equivalence Theory
%
\documentclass[orivec,envcountsame]{llncs}
\usepackage{verbatim}
\usepackage[T1]{fontenc}
\usepackage[latin1]{inputenc}
\usepackage{amsfonts, amssymb, amsmath}

\usepackage{bussproofs}

\newcommand{\Exhibits}[1]{\mathrel{\downarrow_{#1}}}
\newcommand{\ExhibitsA}{\Exhibits{A}}
\newcommand{\ExhibitsT}{\Exhibits{T}}
\newcommand{\ExhibitsE}{\Exhibits{E}}

\newcommand{\Reveals}[1]{\mathrel{\Downarrow_{#1}}}
\newcommand{\RevealsA}{\Reveals{A}}
\newcommand{\RevealsT}{\Reveals{T}}
\newcommand{\RevealsE}{\Reveals{E}}

\newcommand{\DEq}{\bumpeq}
\newcommand{\NotDEq}{\not\bumpeq}
\newcommand{\DEqA}{\mathrel{\DEq_{A}}}
\newcommand{\DEqT}{\mathrel{\DEq_{T}}} 
\newcommand{\DEqE}{\mathrel{\DEq_{E}}} 

\newcommand{\Eq}{\approx}
\newcommand{\NotEq}{\not\approx}
\newcommand{\EqA}{\mathrel{\Eq_{A}}}
\newcommand{\EqT}{\mathrel{\Eq_{T}}} 
\newcommand{\EqE}{\mathrel{\Eq_{E}}}

\newcommand{\LTOC}{\approxeq}

\newcommand{\Means}{\eqdef}
\newcommand{\Does}[1]{\xderives{#1}}
\newcommand{\DoesTaus}{\mathrel{\Rightarrow}}
\newcommand{\Weak}[1]{\stackrel{#1}{\Rightarrow}}
\newcommand{\Theorem}{\vdash}
\newcommand{\OfType}{\mathrel{:}}

% Needed for: \varovee, etc
\usepackage{stmaryrd}

% Needed for: \mathscr{}
\usepackage{mathrsfs}



\newcommand{\shrule}{\vspace{0.5mm}\hrule\vspace{0.5mm}}
\newcommand{\out}{\overline}

%% CaSE macros

\newcommand{\lts}[4]{\langle {#1},{#2},{#3},{#4} \rangle}   

\newcommand{\derives}[1]{\stackrel{#1}{\rightarrow}}
\newcommand{\lderives}[1]{\stackrel{#1}{\longrightarrow}}
\newcommand{\xderives}[1]{\xrightarrow{#1}}
\newcommand{\nderives}[1]{\stackrel{#1}{\nrightarrow}}

\newcommand{\expr}{\text{$\mathcal{E}$}}
\newcommand{\exprb}{\text{$\mathcal{F}$}}
\newcommand{\bouncer}{\text{$\mathcal{F}$}}
\newcommand{\ambop}{\text{$\mathcal{M}$}}
\newcommand{\bambop}{\text{$\mathcal{N}$}}
\newcommand{\procs}{\text{$\mathcal{P}$}}
\newcommand{\timers}{\text{$\mathcal{T}$}}
\newcommand{\labels}{\text{$\mathcal{L}$}}
\newcommand{\names}{\text{$\mathcal{N}$}}
\newcommand{\lowpris}{\text{$\mathcal{C}$}}
\newcommand{\highpris}{\text{$\mathcal{H}$}}
\newcommand{\independent}{\text{$\mathcal{U}$}}
\newcommand{\conames}{\text{$\overline{\names}$}}
\newcommand{\actions}{\text{$\mathcal{A}$}}
\newcommand{\symbols}{\text{$\mathcal{S}$}}

\newcommand{\nil}{\textbf{0}}
\newcommand{\pref}{\,.\,}
\newcommand{\res}[1]{\setminus {#1}}
\newcommand{\hide}[1]{/ \{{#1}\}}
\newcommand{\timeout}[3]{\lfloor{#1}\rfloor {#2} ({#3})}
\newcommand{\stimeout}[3]{\lceil{#1}\rceil {#2} ({#3})}

\newcommand{\eqdef}{\mathrel{~=_\mathrm{def}~}}

\newcommand{\twb}[1]{\approx_{#1}}
\newcommand{\toc}[1]{\approx^{c}_{#1}}

%% operational rules
 
% Derivation Rules
% 1. parameter is name
% 2. parameter is premise
% 3. parameter is conclusion
% 4. parameter is side conditions
%
\newlength{\lrulename}
\setlength{\lrulename}{0.80cm}
 
\newcommand{\Rule}[4]{{#1}&$\displaystyle\frac{#2}{#3}\,{#4}$}
\newcommand{\Rulea}[4]{{#1}\quad$\displaystyle\frac{#2}{#3}\,{#4}$}

% Locality

\newcommand{\loc}[4]{{#1}[ #2 ]^{#3}_{\{#4\}}}
\newcommand{\locv}[4]{{#1}[ #2 ]^{#3}_{#4}}

%% misc macros

\newcommand{\seml}{[\hspace{-0.3ex}[}
\newcommand{\semr}{]\hspace{-0.3ex}]}
\newcommand{\sem}[4]{{}_{#2}\seml #1 \semr^{#3}_{#4}}

\newcommand{\procin}[2]{on\ #1 \tntin{#2}}
\newcommand{\procout}[2]{on\ #1 \tntout{#2}}
\newcommand{\sprocin}[2]{\tntin{#1}\ {#2}}
\newcommand{\sprocout}[2]{\tntout{#1}\ {#2}}
\newcommand{\bin}{\overline{\varovee}}
\newcommand{\bout}{\overline{\varowedge}}
\newcommand{\bopen}{\overline{\varoast}}
\newcommand{\tntin}[1]{\tin #1}
\newcommand{\tntout}[1]{\tout #1}
\newcommand{\tntopen}[1]{\topen #1}
\newcommand{\tin}{\varovee}
\newcommand{\tout}{\varowedge}
\newcommand{\topen}{\varoast}
\newcommand{\mobprim}{\text{$\{ \tin, \tout , \topen\}$}}
\newcommand{\pc}{\mid}


%	mpsmacros.tex
%	Extra commands suggested by mps


\newcommand{\ADDTEXT}{\textit{Text inserted.}}
\newcommand{\DELTEXT}{\textit{Text deleted.}}

% \Proofing{initial}{original text}{suggested replacement}
\usepackage{marginnote}

% Inner paragraphs can't include footnotes. Add the footnote text afterwards, e.g.
%     \InnerProofing{MS}{Old text}{New text in braces}
%     \InsertNewText{}
\newcommand{\TheNewText}{}
\newcommand{\InnerProofing}[3]{%
  $^:$#3\stepcounter{footnote}\footnotemark[\value{footnote}]{}%
  \renewcommand{\TheNewText}{#1: #2}%
  \marginnote{\textbf{*\arabic{footnote}*}}%
}
\newcommand{\InsertNewText}{\footnotetext{\TheNewText}}

% Outer paragraphs are much simpler
\newcommand{\OuterProofing}[3]{%
$^:$#3$^:$\stepcounter{footnote}\marginnote{\textbf{*\arabic{footnote}*}}%
\footnotetext[\value{footnote}]{#1:~#2}%
}

% Don't forget to call \InsertNewText{} right after the inner paragraph ends!
\newcommand{\MPSInner}[2]{\InnerProofing{MS}{#1}{#2}}
\newcommand{\AJHInner}[2]{\InnerProofing{AH}{#1}{#2}}

\newcommand{\MPS}[2]{\OuterProofing{MS}{#1}{#2}}
\newcommand{\AJH}[2]{\OuterProofing{AH}{#1}{#2}}



\begin{document}


\section{Equivalence Theory}
\label{sec:equivalence-theory}

What does it mean for two environs $\locv{m}{P}{B}{\vec\rho}$ and
$\locv{n}{Q}{C}{\vec\sigma}$ to be equivalent? We will adopt the standard Morris-style  \cite{Mor68} (or \emph{may-testing} \cite{DNH84}) approach used to characterise
both equivalence of Mobile Ambients \cite{GC99} and barbed congruence in the
$\pi$-calculus \cite{sangiorgi:book}. Having defined equivalence, we can prove various
safety properties. In particular, we show that $P \Eq Q$ implies that $P$ and
$Q$ have the same type relative to our type system.

\subsection{Immediate Observability}
\label{sec:immediate-observability}

We first define three predicates $\expr \ExhibitsA \alpha$, $\expr \ExhibitsT
\sigma$ and $\expr \ExhibitsE n$, describing the immediate observability of
actions and clocks, and the immediate top-level placement of environs,
respectively. For each predicate $\Exhibits{\bullet}$ we then define a
corresponding contextual equivalence $\Eq_{\bullet}$. Finally, we define
equivalence $\Eq$ for TNT processes to be the conjunction of these three
equivalences. We use the word \emph{symbol} to mean any action, clock or environ
name.

\subsubsection{Actions.}
The notion of immediate observability of an action needs to take account
of possible timeouts. For example, the action $a$ can occur immediately
in both $a\pref\nil$ and $\timeout{a\pref\nil}{\sigma}{b\pref\nil}$,
while the action $b$ can occur immediately in neither. We can capture
the required behaviour quite simply, however, by appealing to our
semantics.

\begin{definition}
We say that $\expr$ \emph{exhibits} the action $\alpha$, and write
$\expr \ExhibitsA \alpha$, provided $\expr \Does{\alpha}$.
\end{definition}


\subsubsection{Clocks.} 

We consider a clock to be immediately observable if it can tick
immediately.

\begin{definition}
We say that $\expr$ \emph{exhibits} the clock $\sigma$, and write $\expr
\ExhibitsT \sigma$, provided $\expr \Does{\sigma}$.
\end{definition}

\subsubsection{Environs.} 

Gordon and Cardelli \cite{GC99} say that an expression $\expr$ exhibits an
environ $n$ provided $n$ occurs at top-level in $\expr$. We adapt this
definition to include information about bouncers and clock sets, and say that
$\expr$ exhibits the environ (actually an environ \emph{context}) $\locv{n}{}{B}{\vec{\sigma}}$
provided this occurs at top-level. However, our definition again needs to
take account of potential time-outs. For example, $\loc{n}{}{B}{\rho}$ is
exhibited in both $\loc{n}{\expr}{B}{\rho}$ and
$\timeout{\loc{n}{\exprb}{B}{\rho}}{\sigma}{\nil}$.

\begin{definition}
We say that $\expr$ \emph{exhibits} the environ $\locv{n}{\,}{B}{\vec{\sigma}}$,
and write $\expr \ExhibitsE \locv{n}{\,}{B}{\vec{\sigma}}$, provided $\expr$ is 
of one of the forms

\begin{enumerate}
\item
    $\locv{n}{\exprb}{B}{\vec{\sigma}}$; or
\item
    $\timeout{\expr' }{\rho}{\exprb}$ or $\stimeout{\expr' }{\rho}{\exprb}$,
    where $\expr' \ExhibitsE \locv{n}{\,}{B}{\vec{\sigma}}$;
\end{enumerate}
\end{definition}


\subsection{Eventual Observability}
\label{sec:eventual-observability}

A symbol $x$ is \emph{eventually observable} (we shall sometimes say it
is \emph{revealed}) in an expression $\expr$ precisely when $x$ is the
first \emph{relevant} action of the appropriate class. As usual, we
write $\DoesTaus$ for the reflexive transitive closure of $\Does{\tau}$, 
$\Weak{\gamma}$ for $\DoesTaus\Does{\gamma}\DoesTaus{\gamma}$ and
$\expr\res{a}$ for $\expr\res{\{a\}}$.


\subsubsection{Actions.}

There is only one unobservable action, $\tau$. This does not mean,
however, that $\tau$ cannot be revealed. Rather, we don't care about
\emph{additional} occurrences of $\tau$.

\begin{definition}
We say that $\expr$ \emph{reveals} the action $\alpha$, and write $\expr
\RevealsA \alpha$, provided there is some $\expr'$ such that $\expr
\DoesTaus \expr' \ExhibitsA \alpha$.
\end{definition}

\subsubsection{Clocks.}

Clocks are used to delimit activities, so we should allow normal actions
to proceed between clock ticks without prejudicing the subsequent tick's
right to be considered eventually observed.

\begin{definition}
We say that $\expr$ \emph{reveals} the clock $\sigma$, and write $\expr
\RevealsT \sigma$, provided there is some $\expr'$ and some $\vec{s} \in
\left( \act \cup \{ \tin, \tout, \topen \}\right)^*$ such that $\expr
\Does{\vec{s}} \expr' \ExhibitsT \sigma$.
\end{definition}


\subsubsection{Environs.}

An environ is eventually observed provided it can be brought to top
level by some sequence of mobility-based reductions.

\begin{definition}
We say that $\expr$ \emph{reveals} the environ $\locv{n}{\,}{B}{\vec{\sigma}}$, and write $\expr
\RevealsE \locv{n}{\,}{B}{\vec{\sigma}}$, provided there is some $\expr'$ and some $\vec{\ambop} \in
\{\tin, \tout, \topen\}^*$ such that $\expr \Does{\vec{\ambop}} \expr'
\ExhibitsE \locv{n}{\,}{B}{\vec{\sigma}}$.
\end{definition}





\subsection{Contextual Congruence}
\label{sec:contextual-congruence}

Before we can define contextual congruence, we need to decide what it is for 
two expressions to reveal the \emph{same} environ. Our definition is
recursive, since the equivalence of $\locv{m}{}{\expr}{\vec{\sigma}}$ and
$\locv{n}{}{\exprb}{\vec{\rho}}$ relies on the equivalence of $\expr$ and
$\exprb$.

As usual, a \emph{context} $C$ is an expression with one or more
\emph{holes}. We write $[\,]$ for a hole, and $C[\expr]$ for the expression
obtained by filling $C$'s holes with copies of $\expr$ (this may result
in free names and variables in $\expr$ becoming bound).

\begin{definition}

A symmetric relation $R \subseteq \proc \times \proc$ is a \emph{contextual
equivalence} if for all contexts $C[\,]$, expressions $\expr$ and $\exprb$,
$\alpha \in \actions$, $\sigma in \timers$, and environs
$\locv{n}{\,}{B}{\vec{\sigma}}$,

\begin{enumerate}
\item
     $C[\expr] \ExhibitsA \alpha \Longrightarrow C[\exprb] \RevealsA \alpha$ and
     vice versa; we say the expressions are \emph{action equivalent}, written 
     $\expr \EqA \exprb$; 
\item
     $C[\expr] \ExhibitsT \sigma \Longrightarrow C[\exprb] \RevealsT \sigma$ and
     vice versa; we say the expressions are \emph{timer equivalent}, written 
     $\expr \EqT \exprb$; 
\item
     they reveal the same environs (up to equivalence of bouncers) in every 
     context:
     \begin{enumerate}
     \item
        If $C[\expr] \RevealsE \locv{n}{\,}{\expr'}{\vec{\sigma}}$
        then 
        $C[\exprb] \RevealsE \locv{n}{\,}{\exprb'}{\vec{\sigma}}$
        for some $\exprb' \Eq \expr'$;
     \item
        If $C[\exprb] \RevealsE \locv{n}{\,}{\exprb'}{\vec{\sigma}}$
        then
        $C[\expr] \RevealsE \locv{n}{\,}{\expr'}{\vec{\sigma}}$
        for some $\expr \Eq \exprb'$.
     \end{enumerate}
     We say the expressions are \emph{environ equivalent}, written $\expr \EqE 
     \exprb$.
\end{enumerate}
\end{definition}


\begin{proposition}
\label{prop:context-substitution} 
Suppose $\expr \Eq \exprb$. Given any context $C[]$, we have $C[\expr]
\Eq C[\exprb]$.
\end{proposition}
\begin{proof}
Given any context $C'[]$, let $C'[C[]]$ be the result of filling the
holes in $C'$ with copies of $C$. Then $C'[C[]]$ is itself a context,
whence $C'[C[\expr]]$ and $C'[C[\exprb]]$ reveal the same actions,
clocks and environs. Since $C'[]$ was arbitrary, we have $C[\expr] \Eq
C[\exprb]$, as claimed.
\qed \end{proof}

\begin{theorem}
Contextual equivalence is a congruence.
\end{theorem}
\begin{proof}
Follows immediately from proposition \ref{prop:context-substitution}.
\qed \end{proof}


Proving that two expressions are not congruent is generally easy. For
example, $a\pref\nil \NotEq b\pref\nil$ because $a\pref\nil$ reveals $a$
but $b\pref\nil$ doesn't. Proving that two expressions are congruent is
more difficult. We illustrate the technique we use by proving the simple
congruence $\left(a\pref\expr \res{a}\right) \Eq \nil$, which is useful
for tidying up expression reductions. Our proof technique relies on the
nature of reductions in contextual expressions.

Suppose the context $C[\,]$ contains $n$ holes; we will write
$C[\,]_1\dots[\,]_n$. Then $C[\expr]$ means $C[\expr]_1\dots[\expr]_n$. If
$\expr \Does{x} \expr'$, then (depending on the nature of $C$), it is possible
that $C[\expr] \Does{x} C[\expr]_1\dots[\expr']_j\dots[\expr]_n$ for some $j$.
We say that such a transition is \emph{local} to $\expr$ in $C$. On the other
hand, there are some transitions that $C[\expr]$ can perform, regardless of
$\expr$. We call these transitions \emph{global} in $C$. For example, the
transition \[ \out{a}.[\,] \mid a \pref b \pref [\,] \quad\Does{\tau}\quad [\,]
\mid b \pref [\,] \] is global. All other transitions in $C[\expr]$ are
\emph{distributed}.

Distributed transitions are those that involve reduction of subcontexts
of the form $C'|[\,]$, $C' + [\,]$ and
$\locv{n}{\,}{B}{\vec{\sigma}}$. For example,
\begin{itemize}
\item
    $\nil + a\pref\nil \Does{\sigma}$ but $\nil + \Delta_\sigma
     \not\Does{\sigma}$, so clock transitions are distributed in the
     context $\nil + [\,]$.
\item
    $a\pref\nil \mid \out{a}\pref\nil \Does{\tau}$ but $a\pref\nil \mid
     \nil \not\Does{\tau}$, so $\Does{\tau}$ is distributed in the
     context $a\pref\nil \mid [\,]$.
\item
    $\locv{m}{ \locv{n}{\tntout{m}\pref\nil}{B}{\vec{\sigma}}}%
     {(\bout\pref\Omega)}{\vec{\rho}} \Does{\tout}$ but
    $\locv{m}{ \nil }{(\bout\pref\Omega)}{\vec{\rho}} \not\Does{\tout}$,
     so $\Does{\tout}$ is distributed in the context $\locv{m}{\,}%
     {(\bout\pref\Omega)}{\vec{\rho}}$.
\end{itemize}
The following results are immediate.

\begin{proposition}
Suppose $\expr \not\Does{x}$ where $x$ is either $a$ or $\sigma$. If
$C[\expr] \Does{x}$ then the $\Does{x}$ transition is global.
\end{proposition}

\begin{proposition}
The only distributed transitions are $\Does{\sigma}$, $\Does{\tau}$,
$\Does{\tin}$, $\Does{\tout}$ and $\Does{\topen}$.
\end{proposition}

We can now prove the congruence claimed above. To see that
$\left(a\pref\expr \res{a}\right) \Eq \nil$, we observe that
$\left(a\pref\expr \res{a}\right)$ cannot reveal any
actions. Consequently, any visible action in $C[a\pref\expr \res{a}]$
must be global, and hence revealed by $C[\nil]$. Moreover, the lack of
visible actions means that $\Does{\tau}$ cannot be generated by
interaction between $a\pref\expr \res{a}$ and a neighbouring
subexpression. Therefore, $\left(a\pref\expr \res{a}\right) \EqA
\nil$. Since neither expression cites environs or mobility transitions,
they cannot contribute to distributed mobility transitions in any
context $C$, whence $\left(a\pref\expr \res{a}\right) \EqE
\nil$. Finally, we note that both expressions reveal all clocks
$\sigma$, so that $\left(a\pref\expr \res{a}\right) \EqT \nil$, and we
are done.


\subsection{Located Temporal Observation Congruence}
\label{sec:ltoc}

Following an idea in \cite{case}, we now use $\Eq$ to define a subcongruence
$\LTOC$ which adds bisimulation propeties to the contextual equivalence already
encapsulated.

\begin{definition}
A symmetric relation $R\mathrel{\subseteq}\mathop{\EqE}$ is a \emph{located temporal
observation equivalence} if for every $(P,Q) \in R$, $\kappa \in \labels \setminus
\timers$ and $\sigma \in \timers$
\begin{enumerate}
\item
    $P \Does{\kappa} P' \Longrightarrow 
       \exists Q' . Q \Weak{\kappa} Q' \mbox{ and } P' \Eq Q'$
\item
    $P \Does{\sigma} P' \Longrightarrow
       \exists Q' . Q \Does{\sigma} Q' \mbox{ and } (P', Q') \in R$
\end{enumerate}
Write $P \LTOC Q$ if $(P,Q) \in R$ for some located temporal observation
equivalence $R$.
\end{definition}

\begin{lemma}
The equivalence $\LTOC$ is a congruence.
\end{lemma}
\begin{proof}

\end{proof}

\bibliographystyle{splncs}
\bibliography{literature}


\end{document}
