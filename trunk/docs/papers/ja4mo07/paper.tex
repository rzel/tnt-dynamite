% sigproc-sp.tex
% A Framework for Mobile Java Applications
% Author: A.J. Hughes
% based upon LaTeX2.09 Guidelines, 9 June 1996
% Revisions:  17th June 2007

\documentclass{acm_proc_article-sp}
% Needed for: \varovee, etc
\usepackage{stmaryrd}

% Needed for: \mathscr{}
\usepackage{mathrsfs}



\newcommand{\shrule}{\vspace{1.0mm}\hrule\vspace{1.0mm}}
\newcommand{\out}{\overline}

%% CaSE macros

\newcommand{\crho}[1]{\rho_{\!#1}}
\newcommand{\csigma}[1]{\sigma_{\!#1}}
\newcommand{\inita}[1]{\mathcal{I\!\!A}({#1})}
\newcommand{\initc}[1]{\mathcal{I\!C}({#1})}

\newcommand{\lts}[4]{\langle {#1},{#2},{#3},{#4} \rangle}   

\newcommand{\derives}[1]{\stackrel{#1}{\rightarrow}}
\newcommand{\lderives}[1]{\stackrel{#1}{\longrightarrow}}
\newcommand{\xderives}[1]{\xrightarrow{#1}}
\newcommand{\nderives}[1]{\stackrel{#1}{\nrightarrow}}
\newcommand{\obsderives}[1]{\stackrel{#1}{\Rightarrow}}    

\newcommand{\expr}{\text{$\mathcal{E}$}}
\newcommand{\exprb}{\text{$\mathcal{F}$}}
\newcommand{\bouncer}{\text{$\mathcal{F}$}}
\newcommand{\ambop}{\text{$\mathcal{M}$}}
\newcommand{\bambop}{\text{$\mathcal{N}$}}
\newcommand{\proc}{\text{$\mathcal{P}$}}
\newcommand{\act}{\text{$\mathcal{A}$}}
\newcommand{\timers}{\text{$\mathcal{T}$}}
\newcommand{\labels}{\text{$\mathcal{L}$}}
\newcommand{\names}{\text{$\mathcal{N}$}}
\newcommand{\lowpri}{\text{$\mathcal{C}$}}
\newcommand{\highpri}{\text{$\mathcal{H}$}}
\newcommand{\independent}{\text{$\mathcal{U}$}}
\newcommand{\conames}{\text{$\overline{\names}$}}
\newcommand{\actions}{\text{$\mathcal{A}$}}

\newcommand{\nil}{\textbf{0}}
\newcommand{\pref}{\,.\,}
\newcommand{\comp}{\;|\;}
\newcommand{\res}[1]{\setminus {#1}}
\newcommand{\hide}[1]{/ {#1}}
\newcommand{\timeout}[3]{\lfloor{#1}\rfloor {#2} ({#3})}
\newcommand{\stimeout}[3]{\lceil{#1}\rceil {#2} ({#3})}
\newcommand{\dottimeout}[5]{\timeout{{#1}}{{#2}}{{#3}}\ldots {#4} ({#5})}
\newcommand{\hid}{/}

\newcommand{\eqdef}{\operatorname{\stackrel{\textrm{def}}{=}}}

\newcommand{\twb}[1]{\approx_{#1}}
\newcommand{\toc}[1]{\approx^{c}_{#1}}

%% operational rules
 
% Derivation Rules
% 1. parameter is name
% 2. parameter is premise
% 3. parameter is conclusion
% 4. parameter is side conditions
%
\newlength{\lrulename}
\setlength{\lrulename}{0.80cm}
 
\newcommand{\Rule}[4]{\makebox[\lrulename]{{\rm #1}\hfill}
                      $\displaystyle\frac{#2}{#3}\,{#4}$}

% Locality

\newcommand{\loc}[4]{{#1}[ #2 ]^{#3}_{\{#4\}}}
\newcommand{\locv}[4]{{#1}[ #2 ]^{#3}_{#4}}
\newcommand{\nloc}[3]{{#1}[ #2 ]^{#3}}
\newcommand{\lcloc}[3]{{#1}[ #2 ]_{\{#3\}}}
\newcommand{\lncloc}[2]{{#1}[ #2 ]}

%% misc macros

\newcommand{\pair}[2]{\langle {#1},{#2} \rangle}
\newcommand{\threetuple}[3]{\langle {#1},{#2},{#3} \rangle}
\newcommand{\df}{\operatorname{=_{\textrm{df}}}}
\newcommand{\seml}{[\hspace{-0.3ex}[}
\newcommand{\semr}{]\hspace{-0.3ex}]}
\newcommand{\semtwo}[5]{{#2}_{#3}\seml #1 \semr^{#4}_{#5}}
\newcommand{\sem}[4]{{}_{#2}\seml #1 \semr^{#3}_{#4}}

\newcommand{\procin}[2]{on\ #1 \tntin{#2}}
\newcommand{\procout}[2]{on\ #1 \tntout{#2}}
\newcommand{\bin}{\overline{\varovee}}
\newcommand{\bout}{\overline{\varowedge}}
\newcommand{\bopen}{\overline{\varoast}}
\newcommand{\tntin}[1]{\tin\!#1}
\newcommand{\tntout}[1]{\varowedge #1}
\newcommand{\tntopen}[1]{\varoast #1}
\newcommand{\tin}{\varovee}
\newcommand{\tout}{\varowedge}
\newcommand{\topen}{\varoast}
\newcommand{\mobprim}{\text{$\{ \tin, \tout , \topen\}$}}
\newcommand{\pc}{\mid}


\begin{document}

\title{A Framework for Mobile Java Applications}
\numberofauthors{1}
\author{
\alignauthor
Andrew Hughes\\
       \affaddr{Department of Computer Science,}
       \affaddr{University of Sheffield}\\
       \affaddr{Regent Court,}
       \affaddr{211 Portobello Street}\\
       \affaddr{Sheffield, UK}\\
       \affaddr{S1 4DP}\\
       \email{andrew@dcs.shef.ac.uk}
}
\maketitle

\begin{abstract}
\end{abstract}

\category{D.3.3}{Programming Languages}{Language Constructs and Features}[Frameworks]
\category{D.1.3}{Programming Techniques}{Concurrent Programming}
\category{D.1.5}{Programming Techniques}{Object-oriented Programming}
%\category{D.3.2}{Programming Languages}{Language Classifications}[Java]
%\category{F.4.3}{Mathematical Logic and Formal Languages}{Formal Languages}[CCS]

\terms{Design}

\keywords{Java, Mobility, Process Calculus, CCS, Ambient Calculus, CaSE,
TNT}

\section{Introduction}

\section{The Theoretical Backbone}

DynamiTE is grounded in the process calculus TNT (\emph{Typed Nomadic
Time}) \cite{hughes:nt, tnt}, which provides a formal abstraction of
concurrent process behaviour.  Using this theoretical framework, a
thread of execution can be described as a series of sequential actions
which incorporates internal behaviour, interprocess communication
(\ref{ipc}), multiprocess synchronisation (\ref{sync}) and mobility
(\ref{structmob}, \ref{procmob}).  TNT utilises and combines
well-established concepts from existing process calculi, including
synchronisation from Milner's CCS \cite{milner:ccs}, mobility from
Cardelli and Gordon's Ambient Calculus \cite{amb} and global discrete
time from Norton, L{\"u}ttgen and Mendler's CaSE \cite{case}.

Take the following example process, $a.\tau.b.\nil$.  The $.$ operator
is used to prefix a process with an action and is used repetitively to
form a complete description of the behaviour of a process.  In a formal
syntactic definition, this is written $a.P$ to denote an action $a$
followed by another process, $P$, which may itself be of the form $a.P$.
This example states that three actions should be performed, $a$, $\tau$
and $b$ before the process evolves into the predefined $\nil$ process,
which represents a process with no explicit behaviour\footnote{It still
exhibits some behaviour, as it can idle over time.}.

We generally classify actions by their observability.  The action with
the name $\tau$ is special, as it denotes arbitrary internal behaviour.
Other actions ($a$ and $b$ in the above example) are observable;
generally, when comparing two processes, it is usual only to match only
on such observable behaviour, and discount internal activity.

Such sequential behaviour can be made more interesting by introducing
non-deterministic behaviour via three control flow operators, $+$,
$\timeout{P}{\sigma}{Q}$ and $\stimeout{P}{\sigma}{Q}$.  The first of
these, $+$, is a basic choice operator inherited from CCS.  When two
processes, $P$ and $Q$, are connected by this operator, a
non-deterministic choice will be made which causes one to execute and
the other to be lost.  From the process, $a.P + b.Q$ either $P$ or $Q$
may result, following the $a$ or $b$ action being performed
respectively.

The other operators interact with TNT's notion of global discrete time,
realised by \emph{clock signals}.  Such signals occur in situations
where they can not be pre-empted by the presence of an internal action.
All other action prefixes and the $\nil$ process can idle, while time
passes.  The two timeout operators allow processes to respond to the
ticks of a specific clock, the left-hand process executing if the clock
does not emit a signal and the right-hand executing if it does.  In
$\timeout{a.P}{\sigma}{b.Q}$, we are left with $b.Q$ if the clock
$\sigma$ ticks.  If it doesn't, then the $a$ action is performed and the
resulting process is simply $P$.

TNT provides two timeout operators, which have different behaviour with
respect to processes idling.  The above example uses the fragile
variant, where, if the left-hand side ($a.P$) can idle over another
clock, the process will evolve into just being that process.  The
alternate stable operator $\stimeout{P}{\sigma}{Q}$ lets the timeout
stay in place; it is only removed when an explicit action from $P$
occurs or $\sigma$ ticks.

A clock may be prevented from ticking by using the $\Delta_\sigma$
operator.  More generally, $\Delta$ stops all clocks.  This transcends
up through the binary operators, $+$ and $\mid$ (introduced below), as
they require the constituent processes to both be able to idle in order
for the resulting process to do so.  For example, $\sigma$ may not tick
over $a.\nil + \Delta_\sigma$.

Finally, a process can also be defined with recursive behaviour.  The
process, $\mu X.a.X$ repeatedly produces $a$ actions.  This is achieved
by the appearance of $\mu X$ which binds the variable, $X$, to the
content appearing after the $X.$.  When $X$ occurs in the process body,
it results in a substitution.  More simply put, when the $a$ action in this
example is performed, the process will become simply $X$.  This is then
replaced by its value, $a.X$, allowing further $a$ actions to occur.

\subsection{Interprocess Communication}
\label{ipc}

The constructs we describe above are fine for specifying sequential
systems, but the main focus of process calculi is to provide an abstract
representation of concurrent systems.  TNT, and its predecessors, make
provision for this via the parallel composition operator, $\mid$.  When
two processes, $P$ and $Q$ are joined with this operator, they are said
to execute concurrently.  

The actual concurrent operation of the two processes is realised through
\emph{interleaving}.  A process $a.P \mid b.Q$ will evolve into one of
two possible processes, $P \mid b.Q$ or $a.P \mid Q$ by performing
either the $a$ or $b$ action respectively.  Thus, the behaviour of
$\mid$ is just like that of $+$, except that, as this represents two
concurrent processes rather than a control branch in a single process,
both sides remain in place.

With this mechanism, we can represent two orthogonal processes running
at the same time and the different permutations of action sequences
possible from applying an interleaved semantics to concurrency.
However, to represent truly interesting behaviour, we need to also
allow the processes to interact.  Action naming again becomes
significant here, as we assume that two processes interact if they emit
a corresponding pair of actions simultaneously.

We've tended to use actions named $a$ and $b$ above.  One of the reasons
for this is that these two actions don't pair up.  An action $a$ can
only pair up with a corresponding co-action, $\overline{a}$.  It is
common to see the name $a$ as being a reference to a channel $a$, with
the action $a$ being an input and $\overline{a}$ being the output.
Indeed, this is how they are used within DynamiTE.

A process such as $a.P \mid \overline{a}.Q$ can evolve to $P \mid
\overline{a}.Q$ or $a.P \mid Q$, just as we saw above.  However, as $a$
and $\overline{a}$ are both available at the same time, the two
processes can synchronise, causing both processes to evolve in one
step to $P \mid Q$.  Such behaviour can be enforced by restricting the
scope of the name $a$.  In the process, $(a.P \mid \overline{a}.Q)
\hide{a}$, actions involving $a$ (both $a$ and $\overline{a}$) can only
be observed within the brackets due to the presence of the restriction
operator, $\hide{a}$.  As a result, the two are forced to synchronise
with each other.

\subsection{Process Synchronisation}
\label{sync}

Synchronisation is a fundamental part of the calculus, and observable
actions, in practice, are used for this purpose.  Thus, actions should
be restricted at appropriate points to enforce this behaviour.  Clearly,
combinatorial explosion may result if restriction is not appropriately
applied, as each pair of actions will produce three alternatives, rather
than one deterministic action.  

Another pertinent point is that synchronisation emits an internal
action.  Recall the behaviour of clocks described above; clock ticks are
pre-empted by such internal actions and so communication also takes
precedence over time progression.  This puts process interaction on an
equal footing with the internal behaviour of a single process.

This synergy of process synchronisation and time is interesting, because
it allows us to effectively detect when all possible interactions have
taken place.  A classic example of this is when a process wants to
broadcast to an arbitrary number of recipients.  How can we construct
such a process, given what we've seen above?

The obvious solution is for the process to output on the channel just
the required number of times.  However, this doesn't give a flexible
solution for an arbitrary number.  Instead of having a general broadcast
agent, we have a process that can transmit to three processes
($\overline{o}.\overline{o}.\overline{o}.P$) and we need a new one for
transmitting to four ($\overline{o}.\overline{o}.\overline{o}.P$).

Alternatively, we can define the process using our recursion operator,
$\mu X.\overline{o}.X$.  This, of course, works for any number of
recipients, but is fundamentally flawed.  What happens when no-one wants
to receive on $o$ any more?  This process will still go on providing an
output; note that there is no $P$ process in this version, because the
process never continues on to do something else.  This is especially
relevant where we wish to use this model as the foundation for a
practical implementation, as this process translates into a thread that
never terminates.

The solution is to utilise the timeout operator to provide a base case
for the recursion.  When $\overline{o}$ can synchronise with a
recipient, $o$, the resulting internal action, $\tau$, will stop the
clock from ticking.  Thus, when the clock does tick, it is indicative of
the fact that no further synchronisations can take place and so our
broadcast agent can go and do something else, which we simply refer to
as $P$.  Formally, this is written as
\begin{displaymath} 
\mu X.\stimeout{\overline{o}.X}{\sigma}{P} \mid o.Q \mid o.R
\end{displaymath}
where $o.Q$ and $o.R$ are two recipients.  We can trivially add more,
due to the way this process is now constructed.  For each recipient, a
synchronisation will take place between it and the broadcast agent.  The
broadcast agent will then recurse, recreating the original situation
with one less recipient.  When there are no further recipients, $\sigma$
will be allowed to tick, causing the broadcast agent to evolve into $P$.

Such a n-ary process synchronisation mechanism is believed to be novel
within the field of mobile process calculi, originating from a
non-mobile process calculus (CaSE) and extended within TNT.

\subsection{Structural Mobility}
\label{structmob}

The interactions described so far are all localised.  Mobility in TNT is
realised by a hierarchy of locations we refer to as \emph{environs}.
Processes reside within these environs, and their interaction is limited
to within their bounds.  Environs also restrict the behaviour of clocks.
Each environ has an associated set of clocks, which can tick within that
environ and any sub-environs.  Outside the environ's bounds, the clock
ticks are transformed into internal actions, which again introduces a
form of precedence.  If a clock hidden inside a environ ticks, it will
prevent the ticks of any clock further up the hierarchy.  

Environs are given a name and a security policy in the form of a special
`bouncer'\footnote{Named after the staff who restrict access to a night
club.  American usage: doorman/woman.} process.  Syntactically, they
appear as
\begin{displaymath}
\loc{m}{\nil}{\bin.\bout.\Omega}{\sigma}
\end{displaymath}
The environ is called $m$\footnote{Names may be of any length, but we
 prefer single letters for formal representations to maintain brevity.
 The same also applies to channel names.} and contains the simple
 process, $\nil$.  The clock $\sigma$ may tick within the bounds of $m$,
 but such ticks appear as internal actions outside.  The sequence
 $\bin.\bout.\Omega$ represents the bouncer process, which restricts the
 usage of mobility primitives with respect to $m$.

These mobility primitives are provided through further syntactic
 constructs, three of which allow the hierarchy to change during
 execution and two which allow processes to move (see \ref{procmob}).
 All five must pair up with a corresponding co-primitive (in much the
 same way as actions match co-actions) provided by the bouncer of the
 environ concerned.  Such behaviour is best demonstrated by example.
 In the following process,
\begin{displaymath}
\locv{n}{\tntin{m}.P}{\Omega}{\emptyset} \mid \loc{m}{\nil}{\bin.\bout.\Omega}{\sigma}
\end{displaymath}
$\tntin{m}$ instructs the surrounding environ $n$ to attempt to move
inside $m$.  It may only do so if the bouncer of $m$ provides the
corresponding co-primitive $\bin$.  This is true in the above, where
$\locv{n}{\tntin{m}.P}{\Omega}{\emptyset}$ may move inside the environ
$m$ and continue as $\locv{n}{P}{\Omega}{\emptyset}$ within this new
environment.  This results in
\begin{displaymath}
\loc{m}{\nil \mid \locv{n}{P}{\Omega}{\emptyset}}{\bout.\Omega}{\sigma}
\end{displaymath}

$\tntout{m}$ provides the opposite behaviour, allowing the surrounding
environ to leave $m$.  If we assume $P$ in the above expands to
$\tntout{m}.P'$, then the process can again interact with the bouncer
and cause $n$ to move outside $m$ to give
\begin{displaymath}
\locv{n}{P'}{\Omega}{\emptyset} \mid \loc{m}{\nil}{\Omega}{\sigma}
\end{displaymath}
which is fairly close to the original process, the exception being that
$\tntin{m}.P$ has evolved into $P'$ and the bouncer has become simply
$\Omega$.  Note that $\Omega$ is the equivalent of $\nil$ for bouncers,
and so no further mobility interactions can involve $m$, making it
immobile (this is the case with the bouncer of $n$ from the start).

Clearly, via these two primitives, the hierarchy may be rearranged
arbitrarily.  The final structural primitive allows environs to be
removed completely\footnote{The opposite of this, creating an environ,
is achieved by simply evolving a process into a new environ
e.g. $a.b.\locv{n}{P}{\Omega}{\sigma}$}.  Again, such an operation must
be permitted by the bouncer of the environ.  This prevents arbitrary
destruction of environs.  Instead, an environ must be defined as
removable on creation.

The bouncer of $m$ in the example below allows $m$ to be destroyed.
When run in parallel with the process $\tntopen{m}.P$, as in
\begin{displaymath}
\tntopen{m}.P \mid \loc{m}{Q}{\bopen.\Omega}{\sigma}
\end{displaymath}
the environ $m$ will disappear and the process inside will enter the
environ above
\begin{displaymath}
P \mid Q
\end{displaymath}
The bouncer of the removed environ is simply lost.  The clock set is
unified with the clock set of the parent environ.  Thus, this has an
effect both on $Q$ (which now executes in a different context) and $P$
(which can now see the ticks of any clock previously hidden in $m$).

\subsection{Process Mobility}
\label{procmob}

The final feature of TNT is process mobility. Unlike the structural
mobility described above, this is \emph{objective}; the process which
exhibits the mobility primitive does not move itself, but instead causes
another process to move.  The migrating process is determined by
matching the action name mentioned in the mobility primitive with one
emitted by another process.  Consider the composition
\begin{displaymath}
  \procin{a}{m}.P \mid a.Q \mid \loc{m}{\nil}{\bin.\bout.\Omega}{\sigma}
\end{displaymath}
where the first process may perform $\procin{a}{m}$, causing the second
to move,
\begin{displaymath}
P \mid \loc{m}{\nil \mid Q}{\bout.\Omega}{\sigma}
\end{displaymath}
its continuation $Q$ now evolving inside the environ $m$.

\emph{Subjective} movement can still be performed by forking a process
in two. For example, suppose $Q$ forks to become $b.Q' \mid
\procout{leave}{m}.\nil$, where the process on the right moves the one
on the left outside $m$. This then allows the inverse operation to be
performed subjectively,
\begin{displaymath}
P \mid Q' \mid \loc{m}{\nil \pc \nil}{\Omega}{\sigma}
\end{displaymath}
to again give a final process which is very similar to the original.

To summarise, the full syntax of TNT is presented below:
\begin{equation*}
  \begin{aligned}
    \expr, \exprb \quad \mathrel{::=} \quad &
      \nil  \mid
      \Omega \mid
      \Delta \mid
      \Delta_{\sigma} \mid
      \alpha . \expr  \mid
      \expr + \exprb \mid
      \expr \mathrel{\!|\!} \exprb \mid
      \timeout{\expr}{\sigma}{\exprb} \mid \\
    & \stimeout{\expr}{\sigma}{\exprb} \mid 
      \mu X . \expr \mid
      X \mid 
      \expr \res{A} \mid
      \locv{m}{\expr}{\exprb}{\vec{\sigma}} \mid
      \ambop . \expr \\
   \ambop \quad \mathrel{::=} \quad & \tntin{m} \mid \tntout{m} \mid \tntopen{m} \mid
      \procin{\beta}{m} \mid \procout{\beta}{m} \mid \\
   & \bin \mid \bout \mid \bopen
   \end{aligned}
\end{equation*}

\section{Mapping Theory to Practicality}

DynamiTE uses the TNT process calculus described above as the basis for
a concurrent object-oriented framework.  Each syntactic construct is
mapped to an appropriate Java class, which provides the required
functionality.  Operation follows a top-down approach; the complete
system is represented by a single instance of one of these classes.  In
most cases, this will be an appropriate operator which composes further
instances as appropriate.

The simplest of these are the representation of $\nil$ and the internal
action, $\tau$.  $\nil$ provides process termination, while $\tau$
allows the user to implement arbitrary sequential behaviour as required,
using individual instances of the class.

Observable actions form part of the channel subsystem, described in
\ref{channels}.  The realisation of the $+$ operator is achieved as a
class which contains a list of subprocesses from which one is chosen is
at random.  Restriction is also simply a case of stopping actions
travelling further up the hierarchy.  More interesting is the $\mid$
operator, which must allow its subprocesses to operate concurrently.

The most obvious way to achieve this is by mapping individual processes
onto Java threads.  This also means that data can be stored with the
process by means of thread-local variables.  However, we are keen to
offer flexibility in how the individual features of the framework are
implemented.  As a result, a Java thread mapping should be only one of
many possible ways of implementing concurrent processing and, where
possible, we abstract away from this as much as possible, allowing it to
be replaced by other implementations at a later date.  For example,
concurrent processing could also be provided by distinct processes
spawned by the VM or a more complex distributed solution may become
apparent.

\subsection{The Channel Abstraction}
\label{channels}

In the same vein, the implementation of synchronisation channels is also
abstracted in such as way as to allow for differing implementations.
Here, the provision of multiple implementations is more prevalent.  At
its simplest, DynamiTE should provide a way of testing TNT processes and
ensuring they perform as expected.  In this respect, the channels need
do nothing more than exist.

More complex solutions are of course possible and are needed to make the
framework both usable and interesting.  Although currently there is no
realisation of data within the formal layer of the calculus, this only
matters to the extent that we wish transmitted data to alter the
constructs themselves via substitution\footnote{The $\pi$ calculus
\cite{milner:pi, picalctutorial} is an obvious example of this
behaviour, which goes to the extreme of not only allowing data to be
transferred but also references to channels which can then later be used
in the language constructs.  This, in essence, provides the form of
mobility presented in the calculus.}.  Data can be transferred between
processes and used within internal actions without having to be
explictly realised at the formal level.

There are a multitude of ways of implementing such behaviour, ranging
from simple mechanisms like files and TCP/IP sockets to more full-blown
interprocess communication protocols such as Java's RMI, CORBA and web
services.  Fortunately, Java already has plenty of support for plugin
based frameworks (imaging and sound already being implemented in this
fashion) and the new \texttt{java.util.ServiceLoader} API provided in
1.6 makes this simpler still.  This means the user will have a choice of
channel implementation, which may even be further extended by their own
or third-party plugins.

While the implementations of the channels themselves can provide the
input and output mechanisms, interoperability between the two has to
take place at a higher level.  Thus, the onus is on the parallel
implementation to co-ordinate the communication between the two, by
virtue of discovering which names are exposed at the point of
composition.  

A possible simplification becomes apparent here, as some implementations
may make use of channel naming.  For example, if the channel name refers
to a host and port for a TCP/IP implementation, then the sender need
only try and connect to see if a recipient is available.  Channel names
are assumed to be unique, so such a mapping is possible.  However, they
are not unique to a particular process, making it perfectly plausible
for the channel name to occur simultaneously on multiple processes.
There is also the issue of whether they can see each other, according to
the constraints of the calculus, so the decision should still be left to
an appropriate parallel composition operator.

\subsection{Signalling}
\label{signalling}

\subsection{Structural Changes}
\label{structchange}

\subsection{Migration}
\label{migration}

\section{Related Work}

\section{Future Work}

\subsection*{Acknowledgements}

This work is supported by a doctoral training award from the Engineering
and Physical Sciences Research Council ({EPSRC}).

\bibliographystyle{abbrv}
\bibliography{literature}
\end{document}
