% sigproc-sp.tex
% A Framework for Mobile Java Applications
% Author: A.J. Hughes
% based upon LaTeX2.09 Guidelines, 9 June 1996
% Revisions:  17th June 2007

\documentclass{sig-alternate}
% Needed for: \varovee, etc
\usepackage{stmaryrd}

% Needed for: \mathscr{}
\usepackage{mathrsfs}



\newcommand{\shrule}{\vspace{0.5mm}\hrule\vspace{0.5mm}}
\newcommand{\out}{\overline}

%% CaSE macros

\newcommand{\lts}[4]{\langle {#1},{#2},{#3},{#4} \rangle}   

\newcommand{\derives}[1]{\stackrel{#1}{\rightarrow}}
\newcommand{\lderives}[1]{\stackrel{#1}{\longrightarrow}}
\newcommand{\xderives}[1]{\xrightarrow{#1}}
\newcommand{\nderives}[1]{\stackrel{#1}{\nrightarrow}}

\newcommand{\expr}{\text{$\mathcal{E}$}}
\newcommand{\exprb}{\text{$\mathcal{F}$}}
\newcommand{\bouncer}{\text{$\mathcal{F}$}}
\newcommand{\ambop}{\text{$\mathcal{M}$}}
\newcommand{\bambop}{\text{$\mathcal{N}$}}
\newcommand{\procs}{\text{$\mathcal{P}$}}
\newcommand{\timers}{\text{$\mathcal{T}$}}
\newcommand{\labels}{\text{$\mathcal{L}$}}
\newcommand{\names}{\text{$\mathcal{N}$}}
\newcommand{\lowpris}{\text{$\mathcal{C}$}}
\newcommand{\highpris}{\text{$\mathcal{H}$}}
\newcommand{\independent}{\text{$\mathcal{U}$}}
\newcommand{\conames}{\text{$\overline{\names}$}}
\newcommand{\actions}{\text{$\mathcal{A}$}}
\newcommand{\symbols}{\text{$\mathcal{S}$}}

\newcommand{\nil}{\textbf{0}}
\newcommand{\pref}{\,.\,}
\newcommand{\res}[1]{\setminus {#1}}
\newcommand{\hide}[1]{/ \{{#1}\}}
\newcommand{\timeout}[3]{\lfloor{#1}\rfloor {#2} ({#3})}
\newcommand{\stimeout}[3]{\lceil{#1}\rceil {#2} ({#3})}

\newcommand{\eqdef}{\mathrel{~=_\mathrm{def}~}}

\newcommand{\twb}[1]{\approx_{#1}}
\newcommand{\toc}[1]{\approx^{c}_{#1}}

%% operational rules
 
% Derivation Rules
% 1. parameter is name
% 2. parameter is premise
% 3. parameter is conclusion
% 4. parameter is side conditions
%
\newlength{\lrulename}
\setlength{\lrulename}{0.80cm}
 
\newcommand{\Rule}[4]{{#1}&$\displaystyle\frac{#2}{#3}\,{#4}$}
\newcommand{\Rulea}[4]{{#1}\quad$\displaystyle\frac{#2}{#3}\,{#4}$}

% Locality

\newcommand{\loc}[4]{{#1}[ #2 ]^{#3}_{\{#4\}}}
\newcommand{\locv}[4]{{#1}[ #2 ]^{#3}_{#4}}

%% misc macros

\newcommand{\seml}{[\hspace{-0.3ex}[}
\newcommand{\semr}{]\hspace{-0.3ex}]}
\newcommand{\sem}[4]{{}_{#2}\seml #1 \semr^{#3}_{#4}}

\newcommand{\procin}[2]{on\ #1 \tntin{#2}}
\newcommand{\procout}[2]{on\ #1 \tntout{#2}}
\newcommand{\sprocin}[2]{\tntin{#1}\ {#2}}
\newcommand{\sprocout}[2]{\tntout{#1}\ {#2}}
\newcommand{\bin}{\overline{\varovee}}
\newcommand{\bout}{\overline{\varowedge}}
\newcommand{\bopen}{\overline{\varoast}}
\newcommand{\tntin}[1]{\tin #1}
\newcommand{\tntout}[1]{\tout #1}
\newcommand{\tntopen}[1]{\topen #1}
\newcommand{\tin}{\varovee}
\newcommand{\tout}{\varowedge}
\newcommand{\topen}{\varoast}
\newcommand{\mobprim}{\text{$\{ \tin, \tout , \topen\}$}}
\newcommand{\pc}{\mid}

\begin{document}

\conferenceinfo{Ja4Mo Workshop/PPPJ 2007,} {Sep. 5--7, 2007, Lisboa,
Portugal.}
\CopyrightYear{2007}
\crdata{978-1-59593-672-1/07/0009}

\title{A Framework for Mobile Java Applications}
\numberofauthors{1}
\author{
\alignauthor
Andrew Hughes\\
       \affaddr{Department of Computer Science,}
       \affaddr{University of Sheffield}\\
       \affaddr{Regent Court,}
       \affaddr{211 Portobello Street}\\
       \affaddr{Sheffield, UK}\\
       \affaddr{S1 4DP}\\
       \email{andrew@dcs.shef.ac.uk}
}
\maketitle

\begin{abstract}
We present the Dynamic Theory Execution (DynamiTE) framework for
 creating concurrent object-oriented applications, with semantics
 grounded in a process calculus.  DynamiTE allows a system to be
 constructed as a series of distinct mobile components called
 \emph{environs} which can change position during execution, and between
 which individual processes can migrate.
\end{abstract}

\category{D.3.3}{Programming Languages}{Language Constructs and Features}[Frameworks]
\category{D.1.3}{Programming Techniques}{Concurrent Programming}
\category{D.1.5}{Programming Techniques}{Object-oriented Programming}
%\category{D.3.2}{Programming Languages}{Language Classifications}[Java]
%\category{F.4.3}{Mathematical Logic and Formal Languages}{Formal Languages}[CCS]

\terms{Design}

\keywords{Java, Mobility, Process Calculus, CCS, Ambient Calculus, CaSE,
TNT}

\section{Introduction}

At an implementation level, concurrent systems tend to be designed in a
very ad-hoc way, resulting in complex concepts such as interprocess
communication and code migration becoming difficult to manage and
control.  DynamiTE provides a framework which abstracts away the
implementation details of such concepts, allowing the programmer to
concentrate instead on a set of simpler constructs grounded in the
formal theory of the process calculus, TNT.  In this paper, we first
present an overview of TNT (section \ref{backbone})\footnote{The reader
is refered to the cited papers for the exact semantics and examples, as
space is limited here}, before looking at how its concepts are
implemented within DynamiTE (section \ref{dynamite}) and closing with a
consideration of related and future work.
 
\section{The Theoretical Backbone}
\label{backbone}

DynamiTE is based on the process calculus TNT (\emph{Typed Nomadic
Time}) \cite{hughes:nt, tnt}, which provides a formal abstraction of
concurrent process behaviour.  Using this theoretical framework, a
thread of execution can be described as a series of sequential actions
which incorporates internal behaviour, interprocess communication
(\ref{ipc}), multiprocess synchronisation (\ref{sync}) and mobility
(\ref{structmob}, \ref{procmob}).  TNT utilises and combines
well-established concepts from existing process calculi, including
synchronisation from Milner's CCS \cite{milner:ccs}, mobility from
Cardelli and Gordon's Ambient Calculus \cite{amb} and global discrete
time from Norton, L{\"u}ttgen and Mendler's CaSE \cite{case}.

Take the following example process, $a.\tau.b.\nil$.  The $.$ operator
is used to prefix a process with an action and is used repetitively to
form a complete description of the behaviour of a process.  In a formal
syntactic definition, this is written $a.P$ to denote an action $a$
followed by another process, $P$, which may itself be of the form $a.P$.
This example states that three actions should be performed, $a$, $\tau$
and then $b$ before the process evolves into the predefined $\nil$
process, which represents a process with no explicit
behaviour\footnote{It still exhibits some behaviour, as it can idle over
time.}.

We generally classify such actions by their observability.  The action
with the name $\tau$ is special, as it denotes arbitrary internal
behaviour.  Other actions ($a$ and $b$ in the above example) are
observable; generally, when comparing two processes, it is usual only to
match on such observable behaviour, and discount internal activity.

Such sequential behaviour can be made more interesting by introducing
non-deterministic behaviour via three control flow operators, $+$,
$\timeout{P}{\sigma}{Q}$ and $\stimeout{P}{\sigma}{Q}$.  The first of
these, $+$, is a basic choice operator inherited from CCS.  When two
processes, $P$ and $Q$, are connected by this operator, a
non-deterministic choice will be made which causes one to execute and
the other to be lost.  From the process, $a.P + b.Q$ either an $a$ or
$b$ action may be performed.  Where the action is $a$, we are left
simply with the process $P$.  Likewise, if $b$ is executed, $Q$ remains.

The other two operators interact with TNT's notion of global discrete
time, realised by \emph{clock signals}.  Such signals are emitted in
situations where they can not be pre-empted by the presence of a
high-priority action such as $\tau$.  All other action prefixes and the
$\nil$ process can idle, while the clock ticks, representing the passage
of time.  The two timeout operators allow processes to respond to the
ticks of a specific clock, the left-hand process executing if the clock
does not emit a signal and the right-hand executing if it does.  In
$\timeout{a.P}{\sigma}{b.Q}$, we are left with $b.Q$ if the clock
$\sigma$ ticks.  If it doesn't, then an $a$ action occurs instead and
the resulting process is simply $P$.

TNT provides two timeout operators, which have different behaviour with
respect to processes idling.  The above example uses the fragile
variant, where, if the left-hand side ($a.P$) can idle over another
clock, the ticks of the other clock are treated like actions and cause
the process to evolve to become $P$.  The alternate stable operator
$\stimeout{P}{\sigma}{Q}$ lets the timeout stay in place; it is only
removed when an explicit action from $P$ occurs or $\sigma$ ticks.

A clock may be prevented from ticking by using the $\Delta_\sigma$
operator.  More generally, $\Delta$ stops all clocks.  This transcends
up through the binary operators, $+$ and $\mid$ (introduced below), as
they require the constituent processes to both be able to idle in order
for the resulting process to do so.  For example, $\sigma$ may not tick
over $a.\nil + \Delta_\sigma$.

Finally, a process can also be defined with recursive behaviour.  The
process, $\mu X.a.X$ repeatedly produces $a$ actions.  This is achieved
by the appearance of $\mu X$ which binds the variable, $X$, to the
content appearing after the $X.$.  When $X$ occurs in the process body,
it results in a substitution.  More simply put, when the $a$ action in
this example is performed, the process will become simply $X$.  This is
then replaced by its value, $a.X$, allowing another $a$ action to occur
and so on.

\subsection{Interprocess Communication}
\label{ipc}

The constructs we describe above are fine for specifying sequential
systems, but the main focus of process calculi is to provide an abstract
representation of concurrent systems.  TNT, and its predecessors, make
provision for this via the parallel composition operator, $\mid$.  When
two processes, $P$ and $Q$ are joined with this operator, they are said
to execute concurrently.  

The actual concurrent operation of the two processes is realised through
\emph{interleaving}.  A process $a.P \mid b.Q$ will evolve into one of
two possible processes, $P \mid b.Q$ or $a.P \mid Q$, by performing
either the $a$ or $b$ action respectively.  Thus, the behaviour of
$\mid$ is just like that of $+$, except that both sides remain in place,
since this represents two concurrent processes rather than a control
branch in a single process.

With this mechanism, we can represent two orthogonal processes running
at the same time and the different permutations of action sequences
possible from applying an interleaved semantics to concurrency.
However, to represent truly interesting behaviour, we need to also
allow the processes to interact.  Action naming again becomes
significant here, as we assume that two processes interact if they emit
a corresponding pair of actions simultaneously.

We've tended to use actions named $a$ and $b$ above.  One of the reasons
for this is that these two actions don't pair up.  An action $a$ can
only pair up with a corresponding co-action, $\overline{a}$.  It is
common to see the name $a$ as being a reference to a channel $a$, with
the action $a$ being an input and $\overline{a}$ being the output.
Indeed, this is how they are used within DynamiTE.

A process such as $a.P \mid \overline{a}.Q$ can evolve to $P \mid
\overline{a}.Q$ or $a.P \mid Q$, just as we saw above.  However, as $a$
and $\overline{a}$ are both available at the same time, the two
processes can synchronise, causing both processes to evolve in one
step to $P \mid Q$.  Such behaviour can be enforced by restricting the
scope of the name $a$.  In the process, $(a.P \mid \overline{a}.Q)
\hide{a}$, actions involving $a$ (both $a$ and $\overline{a}$) can only
be observed within the brackets due to the presence of the restriction
operator, $\hide{a}$.  As a result, the two are forced to synchronise
with each other.

\subsection{Process Synchronisation}
\label{sync}

Synchronisation is a fundamental part of the calculus, and observable
actions, in practise, are used for this purpose.  Thus, actions should
be restricted at appropriate points to enforce this behaviour.  Clearly,
combinatorial explosion may result if restriction is not appropriately
applied, as each pair of actions will produce three alternatives, rather
than one deterministic action.  

Another pertinent point is that synchronisation emits an internal
action.  Recall the behaviour of clocks described above; clock ticks are
pre-empted by such internal actions and so communication also takes
precedence over time progression.  This puts process interaction on an
equal footing with the internal behaviour of a single process.

This synergy of process synchronisation and time is interesting, because
it allows us to effectively detect when all possible interactions have
taken place.  A classic example of this is when a process wants to
broadcast to an arbitrary number of recipients.  How can we construct
such a process, given what we've seen above?

The obvious solution is for the process to output on the channel just
the required number of times.  However, this doesn't give a flexible
solution which can handle an arbitrary number.  Instead of having a
general broadcast agent, we have a process that can transmit to three
processes ($\overline{o}.\overline{o}.\overline{o}.P$) and we need a new
one for transmitting to four
($\overline{o}.\overline{o}.\overline{o}.\overline{o}.P$).

Alternatively, we can define the process using our recursion operator,
$\mu X.\overline{o}.X$.  This, of course, works for any number of
recipients, but is fundamentally flawed.  What happens when no-one wants
to receive on $o$ any more?  This process will still go on providing an
output; note that there is no $P$ process in this version, because the
process never continues on to do something else.  In a practical
implementation, this corresponds to a thread that never terminates.

The solution is to utilise the timeout operator to provide a base case
for the recursion.  When $\overline{o}$ can synchronise with a
recipient, $o$, the resulting internal action, $\tau$, will stop the
clock from ticking.  Thus, when the clock does tick, it demonstrates
that no further synchronisations can take place and so our broadcast
agent can go and do something else, which we simply refer to as $P$.
Formally, this is written as
\begin{displaymath} 
\mu X.\stimeout{\overline{o}.X}{\sigma}{P} \mid o.Q \mid o.R
\end{displaymath}
where $o.Q$ and $o.R$ are two recipients.  We can trivially add more, as
prior knowledge of the recipients is no longer required.  If any process
is listening on $o$, a synchronisation will take place between it and
the broadcast agent.  The broadcast agent will then recurse, recreating
the original situation with one less recipient.  When there are no
further recipients, $\sigma$ will be allowed to tick, causing the
broadcast agent to evolve into $P$.

Such a n-ary process synchronisation mechanism is believed to be novel
within the field of mobile process calculi, originating from a
non-mobile process calculus (CaSE) and being extended within TNT.

\subsection{Structural Mobility}
\label{structmob}

The interactions described so far are all localised.  Mobility in TNT is
realised by a hierarchy of locations we refer to as \emph{environs}.
Processes reside within these environs, and their interaction is limited
to within their bounds.  Environs also restrict the behaviour of clocks.
Each environ has an associated set of clocks, which can tick within that
environ and any sub-environs.  Outside the environ's bounds, the clock
ticks are transformed into internal actions, which then pre-empt the
ticks of any clock further up the hierarchy.

Environs are given a name and a security policy in the form of a special
`bouncer'\footnote{Named after the staff who restrict access to a night
club.  American usage: doorman/woman.} process.  Syntactically, they
appear as
\begin{displaymath}
\loc{m}{\nil}{\bin.\bout.\Omega}{\sigma}
\end{displaymath}
The environ is called $m$\footnote{Names may be of any length, but we
 prefer single letters for formal representations to maintain brevity.
 The same also applies to channel names.} and contains the simple
 process, $\nil$.  The clock $\sigma$ may tick within the bounds of $m$,
 but such ticks appear as internal actions outside.  The sequence
 $\bin.\bout.\Omega$ represents the bouncer process, which restricts the
 usage of mobility primitives with respect to $m$.

These mobility primitives are provided through further syntactic
 constructs, three of which allow the hierarchy to change during
 execution and two that allow processes to move (see \ref{procmob}).
 All five must pair up with a corresponding co-primitive (in much the
 same way as actions match co-actions) provided by the bouncer of the
 environ concerned.  In doing so, they emit a high-priority action,
 which, like internal actions, pre-empts clock ticks.  This allows
 mobility primitives to be used in a broadcast style, in the same way as
 we used actions in section \ref{sync}.

Such behaviour is best demonstrated by example.  In the following
 process,
\begin{displaymath}
\locv{n}{\tntin{m}.P}{\Omega}{\emptyset} \mid \loc{m}{\nil}{\bin.\bout.\Omega}{\sigma}
\end{displaymath}
$\tntin{m}$ instructs the surrounding environ $n$ to attempt to move
inside its sibling, $m$. It may only do so if the bouncer of $m$
provides the corresponding co-primitive $\bin$.  This is true in the
above, where $\locv{n}{\tntin{m}.P}{\Omega}{\emptyset}$ may move inside
the environ $m$ and continue as $\locv{n}{P}{\Omega}{\emptyset}$ within
this new environment.  This results in
\begin{displaymath}
\loc{m}{\nil \mid \locv{n}{P}{\Omega}{\emptyset}}{\bout.\Omega}{\sigma}
\end{displaymath}

$\tntout{m}$ provides the opposite behaviour, allowing the surrounding
environ to leave $m$, a parent environ.  If we assume $P$ in the above
expands to $\tntout{m}.P'$, then the process can again interact with the
bouncer and cause $n$ to move outside $m$, giving
\begin{displaymath}
\locv{n}{P'}{\Omega}{\emptyset} \mid \loc{m}{\nil}{\Omega}{\sigma}
\end{displaymath}
which is fairly close to the original process, the exception being that
$\tntin{m}.P$ has evolved into $P'$ and the bouncer has become simply
$\Omega$.  Note that $\Omega$ is the equivalent of $\nil$ for bouncers,
and so no further mobility interactions can involve $m$, making it
immobile (this is the case with the bouncer of $n$ from the start).

Clearly, via these two primitives, the hierarchy may be rearranged
arbitrarily.  The final structural primitive allows environs to be
removed completely\footnote{The opposite of this, creating an environ,
is achieved by simply evolving a process into a new environ
e.g. $a.b.\locv{n}{P}{\Omega}{\sigma}$}.  Again, such an operation must
be permitted by the bouncer of the environ.  This prevents arbitrary
destruction of environs.  Instead, an environ must effectively be
defined as removable on creation.

A bouncer exhibiting the co-action $\bopen$ allows an environ to be
destroyed.  When an ambient with such a bouncer is run in parallel with
the process $\tntopen{m}.P$, as in
\begin{displaymath}
\tntopen{m}.P \mid \loc{m}{Q}{\bopen.\Omega}{\sigma}
\end{displaymath}
the environ $m$ will disappear and the process inside will enter the
environ above giving
\begin{displaymath}
P \mid Q
\end{displaymath}
The bouncer of the removed environ is simply lost.  The clock set is
unified with the clock set of the parent environ, so the operation
effects both $Q$ (now executing in a different context) and $P$ (which
can now see the ticks of any clock previously hidden inside $m$).

\subsection{Process Mobility}
\label{procmob}

The final feature of TNT is process mobility. Unlike the structural
mobility described above, this is \emph{objective}; the process which
exhibits the mobility primitive does not move itself, but instead causes
another process to move.  The migrating process is determined by
matching the action name mentioned in the mobility primitive with one
emitted by another process.  Consider the composition
\begin{displaymath}
  \procin{a}{m}.P \mid a.Q \mid \loc{m}{\nil}{\bin.\bout.\Omega}{\sigma}
\end{displaymath}
where the first process may perform $\procin{a}{m}$, causing the second
to move,
\begin{displaymath}
P \mid \loc{m}{\nil \mid Q}{\bout.\Omega}{\sigma}
\end{displaymath}
its continuation $Q$ now evolving inside the environ $m$.

\emph{Subjective} movement can still be performed by forking a process
in two. For example, suppose $Q$ diverges to become $b.Q' \mid
\procout{leave}{m}.\nil$, where the process on the right moves the one
on the left outside $m$. This then allows the inverse operation to be
performed subjectively,
\begin{displaymath}
P \mid Q' \mid \loc{m}{\nil \pc \nil}{\Omega}{\sigma}
\end{displaymath}
to again give a final process which is very similar to the original.

To summarise, the full syntax of TNT is presented below:
\begin{equation*}
  \begin{aligned}
    \expr, \exprb \quad \mathrel{::=} \quad &
      \nil  \mid
      \Omega \mid
      \Delta \mid
      \Delta_{\sigma} \mid
      \alpha . \expr  \mid
      \expr + \exprb \mid
      \expr \mathrel{\!|\!} \exprb \mid
      \timeout{\expr}{\sigma}{\exprb} \mid \\
    & \stimeout{\expr}{\sigma}{\exprb} \mid 
      \mu X . \expr \mid
      X \mid 
      \expr \res{A} \mid
      \locv{m}{\expr}{\exprb}{\vec{\sigma}} \mid
      \ambop . \expr \\
   \ambop \quad \mathrel{::=} \quad & \tntin{m} \mid \tntout{m} \mid \tntopen{m} \mid
      \procin{\beta}{m} \mid \procout{\beta}{m} \mid \\
   & \bin \mid \bout \mid \bopen
   \end{aligned}
\end{equation*}

\section{Mapping Theory to Practicality}
\label{dynamite}

DynamiTE uses the TNT process calculus described above as the basis for
a concurrent object-oriented framework.  Within this framework,
developers can create concurrent applications simply by implementing the
specific behaviour they require in appropriate subclasses.  Each
syntactic construct is mapped to an appropriate Java class, which
provides the required functionality and relates to others via a common
\texttt{Process} superclass.  Operation follows a top-down approach; the
complete system is represented by a single instance of one of these
classes which, in most cases, will be an operator that composes together
further instances as appropriate.

The simplest \texttt{Process} subclass is the representation of $\nil$,
realised as a class \texttt{Nil} which provides process termination.
The internal action $\tau$ is realised as an abstract class \texttt{Tau}
and this is where the user can implement arbitrary sequential behaviour
as required, by providing a subclass. The observable actions form part
of the channel subsystem, described in \ref{channels}.

The $+$ operator is implemented as a class which contains a list of
subprocesses from which one is chosen at random.  The action to perform
is computed by traversing the hierarchy, so restriction is simply a
matter of providing appropriate filtering, thus preventing the
restricted names from travelling further up the hierarchy.

More interesting is the \texttt{Par} class which implements the $\mid$
operator, as it must allow its subprocesses to operate concurrently.
The most obvious way to achieve this is by mapping individual processes
onto Java threads.  This also means that data can be stored with the
process by means of thread-local variables.  However, we are keen to
offer flexibility in how the individual features of the framework are
implemented.  Java thread mapping is only one way in which concurrent
processing may be implemented and so we abstract away \texttt{Par} from
the threading implementation as much as possible, thus allowing it to be
replaced by other implementations at a later date.  For example,
concurrent processing could also be provided by distinct processes
spawned by the VM or a more complex distributed solution may become
apparent.

\subsection{The Channel Abstraction}
\label{channels}

In the same vein, the implementation of synchronisation channels is
abstracted in such as way as to allow for differing implementations.
Here, the provision of multiple implementations is more prevalent and so
a plugin mechanism is already present.  Fortunately, Java already has
plenty of support for plugin based frameworks (imaging and sound already
being implemented in this fashion) and the new
\texttt{java.util.ServiceLoader} API provided in 1.6 makes this simpler
still.  This allows the user to have freedom of choice with respect to
their chosen channel implementation, which may even be further extended
by their own or third-party plugins.

At its simplest, DynamiTE provides a way of testing TNT processes and
ensuring they perform as expected.  In this respect, the simplest
channel plugin is a dummy channel, which need do nothing more than
simply exist.  More complex solutions are of course possible and are
needed to make the framework both usable and interesting.  

Although currently there is no realisation of data within the formal
layer of the calculus, this only matters to the extent that we wish
transmitted data to alter the constructs themselves via
substitution\footnote{The $\pi$ calculus \cite{picalctutorial} is an
obvious example of such behaviour, which goes to the extreme of not only
allowing data to be transferred but also references to channels which
can then later be used in the language constructs.  This, in essence,
provides the form of mobility present in the $\pi$ calculus.}.  Data can
be transferred between processes and used within internal actions
without having to be explicitly realised at the formal level.  There are
a multitude of ways of implementing data transfer, ranging from simple
mechanisms like files and sockets to more full-blown interprocess
communication protocols such as Java's Remote Method Invocation (RMI),
the Common Object Request Broker Architecture (CORBA) and web services.
The plugin nature of the channel architecture means that any of these
possibilities may be used and more besides.

While the implementations of the channels themselves can provide the
input and output mechanisms, interoperability between the two has to
take place at a higher level.  Thus, the onus is on the parallel
implementation, \texttt{Par}, to co-ordinate the communication between
the two, by virtue of discovering which names are exposed at the point
of composition.

A possible simplification becomes apparent here, as some implementations
may make use of channel naming.  For example, if the channel name refers
to a host and port for a TCP/IP implementation, then the sender need
only try and connect to see if a recipient is available.  Channel names
are assumed to be unique, so such a mapping is possible.  However, they
are not unique to a particular process, making it perfectly plausible
for the channel name to occur simultaneously on multiple processes and
thus for a competition to occur.  There is also the issue of whether
they can actually `see' each other, according to the constraints of the
calculus, so the decision should still be left to an appropriate
parallel composition operator.

\subsection{Signalling}
\label{signalling}

One of the most interesting parts of the DynamiTE framework is the
implementation of clock signals.  While there have been other attempts
to produce frameworks or languages based on process calculi (see section
\ref{relatedwork}), we believe that the rendering of discrete time into
such a context is novel.

The first question to answer when attempting to perform such a
translation is where to actually locate the clocks.  Within TNT, the
obvious answer is within each environ, as these are responsible for
providing the division between processes which can observe clock ticks
and those which can not.  For example, the following environ
\begin{displaymath}
\loc{m}{P}{\Omega}{\sigma}
\end{displaymath}
would be realised as an instance of the \texttt{Environ} class with the
name $m$.  This instance would maintain a reference to the process $P$
with which it interacts.  Not only is the execution of $P$ controlled by
the environ (as with the implementations of $+$ and $\mid$ above), but
it also controls when and how the ticks of $\sigma$ reach $P$.

Recall our earlier description of the calculus, where we mentioned how
clock ticks are always pre-empted by high priority actions, which may
arise either from explicit internal actions denoted by $\tau$, implicit
internal actions caused by synchronisation or movement.  So, in order
for the environ to know whether to propagate a clock tick to the
process, it must first probe it to find out whether such a high priority
action is pending.  Clock ticks may also be prevented by the $\Delta$
and $\Delta_sigma$ constructs, so these must also be checked for.

Both can actually be achieved in one transaction by making the probe the
clock tick.  The clock tick is sent down the process hierarchy until it
reaches a point at which a decision can be made as to whether the tick
should occur or not.  If the tick can occur, it is propagated back up
the hierarchy, eventually stopping when it reaches its host environ
again.  The host environ can be determined by the set of clocks
associated with each environ, which is also used to calculate the
signals to be propagated initially.  If the clock is not allowed to
tick, then the actual action performed is sent instead.

This algorithm is best explained by a couple of prototypical examples.
First, consider 
\begin{displaymath}
\loc{m}{a.\nil + b.\nil}{\Omega}{\sigma}
\end{displaymath}
where the process inside $m$ has no $\tau$ actions, synchronisations,
mobility or clock stop operators, and thus clearly allows the clock
$\sigma$ to tick.  The environ $m$ iterates over its set of clocks (here
just $\sigma$), and sends a tick from each to its process, $a.\nil +
b.\nil$.

This process is realised by an instance of the \texttt{Sum} class, which
composes the two processes together.  A clock can only tick over the
summation operator if it can tick over both sides, so the result from
this instance is simply the result of combining the return value from
probing each of the constituent processes.

Both $a.\nil$ and $b.\nil$ are implemented using instances of the
\texttt{Prefix} class, which composes a \texttt{Channel}\footnote{An
abstract class, instances of which are provided by the channel
architecture described in \ref{channels}} or \texttt{Tau} instance
(unified by the \texttt{Action} class) with another instance of a
\texttt{Process} subclass.  In determining whether a clock can tick, it
first checks that the action is a channel rather than a \texttt{Tau}
instance (which would pre-empt the clock), and then probes the
\texttt{Process} instance.  In both these simple cases, this is an
instance of \texttt{Nil}, which allows clock ticks.

Having determined that the clock may tick, each nested call returns with
the $\sigma$ clock tick, thus propagating it up to the original call in
the environ $m$.  Having seen how this operates for a process that can
tick, it is simple to see how it differs when something prevents the
clock from ticking.  If any part of the query returns something other
than a clock tick, this will be propagated upwards in preference.

Consider what happens if $a.\nil$ is changed to $\tau.\nil$.  The
left-hand side of the summation will receive the $\tau$ action from the
\texttt{Prefix} instance, which then takes priority over the $\sigma$
from the right-hand side and is propagated to the environ, $m$.  This is
the case in any situation where the $\sigma$ is required to compete
against an action, a $\tau$ or a mobility primitive.  The clock stop
operators behave slightly differently in that they don't replace the
action, but instead mark the $\sigma$ action as \emph{stopped}.

Note that a similar method of determining the presence of clock ticks
must take place to handle the \texttt{STimeout} and \texttt{FTimeout}
classes.  Both sides of the timeout are inspected, and behaviour
determined as follows:
\begin{enumerate}
\item If the left-hand side can perform a high-priority action, it will
      be allowed to proceed and the right-hand side need not be
      considered.
\item Otherwise, the possible actions include unpaired actions (such as
      $a$ and $b$) and clock ticks (both from the clock involved in the
      timeout and from other clocks), one of which is chosen to be
      performed.
\item Once the chosen action has been performed, the timeout instance
      will be replaced as appropriate (see section \ref{backbone}). 
\end{enumerate}

\subsection{Structural Changes}
\label{structchange}

The \texttt{Environ} class also places a central role in providing
system structure.  In section \ref{structmob}, we described how
processes are organised into environs and the way communication is
limited to its bounds.  Within DynamiTE, one possible use of environs
is to map them to physical or virtual hosts.  While a simple testing
solution can execute the entire system on a single platform, environs
provide a natural form of process distribution which can be leveraged by
the framework.

This does however give the initial impression that structural mobility
will become very inefficient, if hosts are expected to interact to
determine the feasibility of a move and then actually change position
during execution.  In reality, these issues are minimal.  An inward
movement is always in relation to a sibling, while an outward movement
concerns some parent environ.  As the structure of environs is expected
to closely match the actual physical structure of the hosts, such
interactions should be relatively low cost to perform.  Also, a
structural movement does not change the contents of the moving environ,
only its context.  Thus, only later communication with surrounding
environs is affected.  For example, it may have been able to see a
sibling environ before the movement, but is now inside this environ and
can receive clock ticks emitted by it.

If hosts do not physically move, then what is the point in allowing such
structural changes?  The change in clock signalling just mentioned is
one effect.  In addition, we also make provision for contextual data to
be stored at the environ level, in addition to that stored local to a
particular thread, and transferred via channels.  This gives additional
purpose to the use of structural mobility and process migration, which
we describe next.

\subsection{Migration}
\label{migration}

The final aspect of DynamiTE that we describe here is the migration of a
process from one environ to another, which occurs both as a result of
using one of the process mobility operators and from the behaviour of
$\tntopen$.  This is perhaps one of the most interesting aspects, as it
represents the movement of code from one environment to another,
possibly located in a different physical location.

Migrating an active process is not a simple operation.  Not only must
any remaining code to be executed be transferred, but any local data
must also migrate.  TNT does allow us to achieve a significant amount of
simplification here.  The transferred process is already separated from
other code within the system by virtue of the moving process being in
the form of a \texttt{Prefix} instance.  When the action is matched to
the one used for the mobility operation, the \texttt{Process} instance
is transferred to its new location.  There is no necessity to deal with
code that is currently being executed.

As with concurrency and channel operation, how movement is achieved is
designed to be flexible, with provision being made for distribution and
code migration to be implemented in different ways.  One of the most
obvious ways is to serialise the \texttt{Process} instance and
reconstitute it at its destination.  Migrating a process should then
just be a case of transmitting the serialised instance, followed by any
local data, and beginning execution at the destination.  However, this
is one area in which we expect further study of the existing literature
to enlighten us with more sophisticated ways of achieving such
migration.

\section{Related Work}
\label{relatedwork}

There has already been a significant body of research into providing
concurrent frameworks, including those based on process calculi.
However, we believe our work to be novel in approaching the
implementation of both global discrete time, via clock signalling, and
mobility.

The $\pi$ calculus has been the subject of much of this work, primarily
due to its status as the most prevalant mobile process calculus.  Obliq
\cite{obliq} and Pict \cite{daveturner:phd} are both programming
languages with semantics founded in the $\pi$ calculus, while Nomadic
Pict \cite{wojciechowski:phd} takes this further, introducing
distribution not usually present in the $\pi$ calculus.  Within research
related to the ambient calculus, a machine framework (PAN
cite{sangiorgi:safeambientsmachine}) has been developed and
implemented.  Process calculi, such as the Seal calculus \cite{seal}
have also been developed specifically to provide a formal framework for
a distributed implementation.
%The $\pi$ calculus has been the subject of much of this work, primarily
%due to its status as the most prevalent mobile process calculus.  Obliq
%\cite{obliq} and Pict \cite{daveturner:phd} are both programming
%languages with semantics founded in the $\pi$ calculus, while a machine
%framework (PAN \cite{sangiorgi:safeambientsmachine}) has been developed
%and implemented for the ambient calculus.

\section{Conclusions and Future Work}

To conclude, we have presented the overall structure of the DynamiTE
framework for concurrent systems, with particular note to its more
interesting aspects involving the use of signalling (via clock ticks)
and process migration.  We have also outlined its underlying theoretical
basis in the form of the process calculus TNT, further details of which
are provided in the cited references.

We believe that the framework provides a unique way of developing
concurrent systems.  It provides features which have already proved
advantageous in a theoretical setting, such as the n-ary process
synchronisation mechanism described in \ref{sync}.  The existence of a
formal theory for DynamiTE's behaviour gives many advantages over more
ad-hoc approaches, allowing the underlying mechanisms to be rigorously
examined before being applied to the implementation.  For example, the
equivalence of two processes may be established clearly and
unambiguously in the underlying process calculus and then used to
compare the implementation of a process to its specification.

The DynamiTE framework is still in heavy development.  At its lowest
level, it provides a means of simulating the operations of the TNT
process calculus, allowing them to be more clearly understood.  In
application, it can provide a useful mechanism for structuring
concurrent programs, clearly dividing internal behaviour and
interprocess communication.  The presence of signalling and code
migration also means that fairly complex concepts can be leveraged by
the programmer in the simple manner provided by the framework.

There are still areas we wish to explore in the future.  One such
proposition is the addition of data to the clock signals,
\balancecolumns \noindent allowing them
not only to act as phasing signals but also as a mechanism for broadcast
data.  It would also be interesting to further expand on the plugin
frameworks mentioned, by providing more complex implementations such as
interprocess communication via web services.

\subsection*{Acknowledgements}

This work is supported by a doctoral training award from the Engineering
and Physical Sciences Research Council ({EPSRC}).

\bibliographystyle{abbrv}
\begin{thebibliography}{10}

\bibitem{amb}
L.~Cardelli and A.~D. Gordon.
\newblock {Mobile Ambients}.
\newblock In {\em Proc. of the 1st Intl. Conference on Foundations of Software
  Science and Computation Structures ({FoSSaCS}~'98)}, volume 1378 of {\em
  Lecture Notes in Computer Science}, pages 140--155. Springer, 1998.

\bibitem{hughes:nt}
A.~Hughes.
\newblock {Nomadic Time (Extended Abstract)}.
\newblock In R.~Schmidt and G.~Struth, editors, {\em Proc. of the PhD Programme
  at Relational Methods in Computer Science/Applications of Kleene Algebra
  ({RelMiCS/AKA}) 2006}, number CS-06-09 in University of Sheffield Technical
  Reports, pages 60--64, 2006.

\bibitem{tnt}
A.~Hughes.
\newblock {Timed Mobile Systems}.
\newblock Technical Report CS-07-09, University of Sheffield, 2007.

\bibitem{obliq}
M.~Merro, J.~Kleist, and U.~Nestmann.
\newblock {Mobile Objects as Mobile Processes}.
\newblock {\em Information and Computation}, 177:195--241, 2002.

\bibitem{milner:ccs}
R.~Milner.
\newblock {\em {Communication and Concurrency}}.
\newblock Prentice-Hall, London, 1989.

\bibitem{picalctutorial}
R.~Milner, J.~Parrow, and D.~Walker.
\newblock A calculus of mobile processes, parts {I} and {II}.
\newblock Technical Report ECS-LFCS-89-86, University of Edinburgh, June 1989.

\bibitem{case}
B.~Norton, G.~L{\"u}ttgen, and M.~Mendler.
\newblock {A Compositional Semantic Theory for Synchronous Component-Based
  Design}.
\newblock In {\em Proc. of the 14th Intl. Conference on Concurrency Theory
  ({CONCUR}~'03)}, number 2761 in LNCS, pages 461--476. Springer, 2003.

\bibitem{daveturner:phd}
D.~N. Turner.
\newblock {\em {The Polymorphic Pi-calculus: Theory and Implementation}}.
\newblock PhD thesis, The University of Edinburgh, 1996.

\bibitem{seal}
J.~Vitek and G.~Castagna.
\newblock {Seal: A Framework for Secure Mobile Computations}.
\newblock In {\em Proc. of the {ICCL}~'98 Workshop on Internet Programming
  Languages}, volume 1686 of {\em Lecture Notes in Computer Science}, pages
  47--77. Springer, 1999.

\bibitem{wojciechowski:phd}
P.~T. Wojciechowski.
\newblock {\em {Nomadic Pict: Language and Infrastructure Design for Mobile
  Computation}}.
\newblock PhD thesis, The University of Cambridge, Mar. 2000.

\end{thebibliography}

\end{document}
