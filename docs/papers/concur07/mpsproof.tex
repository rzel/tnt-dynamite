%
%   Timed Mobile Systems
%   Andrew Hughes and Mike Stannett
%   Basic Equivalence Theory
%
\documentclass[orivec,envcountsame]{llncs}
\usepackage{verbatim}
\usepackage[T1]{fontenc}
\usepackage[latin1]{inputenc}
\usepackage{amsfonts, amssymb, amsmath}

\usepackage{bussproofs}

\newcommand{\Exhibits}[1]{\mathrel{\downarrow_{#1}}}
\newcommand{\ExhibitsA}{\Exhibits{\mathscr{A}}}
\newcommand{\ExhibitsC}{\Exhibits{\mathscr{C}}}
\newcommand{\ExhibitsE}{\Exhibits{\mathscr{E}}}

\newcommand{\Reveals}[1]{\mathrel{\Downarrow_{#1}}}
\newcommand{\RevealsA}{\Reveals{\mathscr{A}}}
\newcommand{\RevealsC}{\Reveals{\mathscr{C}}}
\newcommand{\RevealsE}{\Reveals{\mathscr{E}}}

%\newcommand{\Invokes}[1]{\mathrel{\Downarrow_{#1}}}
%\newcommand{\InvokesA}{\Invokes{\mathscr{A}}}
%\newcommand{\InvokesC}{\Invokes{\mathscr{C}}}
%\newcommand{\InvokesE}{\Invokes{\mathscr{E}}}

\newcommand{\Eq}{\approx}
\newcommand{\EqA}{\mathrel{\Eq_{\mathscr{A}}}}
\newcommand{\EqC}{\mathrel{\Eq_{\mathscr{C}}}}
\newcommand{\EqE}{\mathrel{\Eq_{\mathscr{E}}}}

\newcommand{\Means}{\eqdef}
\newcommand{\Does}[1]{\xderives{#1}}
\newcommand{\DoesMany}[1]{\mathrel{{\xderives{#1}}^*}}

% Needed for: \varovee, etc
\usepackage{stmaryrd}

% Needed for: \mathscr{}
\usepackage{mathrsfs}



\newcommand{\shrule}{\vspace{1.0mm}\hrule\vspace{1.0mm}}
\newcommand{\out}{\overline}

%% CaSE macros

\newcommand{\crho}[1]{\rho_{\!#1}}
\newcommand{\csigma}[1]{\sigma_{\!#1}}
\newcommand{\inita}[1]{\mathcal{I\!\!A}({#1})}
\newcommand{\initc}[1]{\mathcal{I\!C}({#1})}

\newcommand{\lts}[4]{\langle {#1},{#2},{#3},{#4} \rangle}   

\newcommand{\derives}[1]{\stackrel{#1}{\rightarrow}}
\newcommand{\lderives}[1]{\stackrel{#1}{\longrightarrow}}
\newcommand{\xderives}[1]{\xrightarrow{#1}}
\newcommand{\nderives}[1]{\stackrel{#1}{\nrightarrow}}
\newcommand{\obsderives}[1]{\stackrel{#1}{\Rightarrow}}    

\newcommand{\expr}{\text{$\mathcal{E}$}}
\newcommand{\exprb}{\text{$\mathcal{F}$}}
\newcommand{\bouncer}{\text{$\mathcal{F}$}}
\newcommand{\ambop}{\text{$\mathcal{M}$}}
\newcommand{\bambop}{\text{$\mathcal{N}$}}
\newcommand{\proc}{\text{$\mathcal{P}$}}
\newcommand{\act}{\text{$\mathcal{A}$}}
\newcommand{\timers}{\text{$\mathcal{T}$}}
\newcommand{\labels}{\text{$\mathcal{L}$}}
\newcommand{\names}{\text{$\mathcal{N}$}}
\newcommand{\lowpri}{\text{$\mathcal{C}$}}
\newcommand{\highpri}{\text{$\mathcal{H}$}}
\newcommand{\independent}{\text{$\mathcal{U}$}}
\newcommand{\conames}{\text{$\overline{\names}$}}
\newcommand{\actions}{\text{$\mathcal{A}$}}

\newcommand{\nil}{\textbf{0}}
\newcommand{\pref}{\,.\,}
\newcommand{\comp}{\;|\;}
\newcommand{\res}[1]{\setminus {#1}}
\newcommand{\hide}[1]{/ {#1}}
\newcommand{\timeout}[3]{\lfloor{#1}\rfloor {#2} ({#3})}
\newcommand{\stimeout}[3]{\lceil{#1}\rceil {#2} ({#3})}
\newcommand{\dottimeout}[5]{\timeout{{#1}}{{#2}}{{#3}}\ldots {#4} ({#5})}
\newcommand{\hid}{/}

\newcommand{\eqdef}{\operatorname{\stackrel{\textrm{def}}{=}}}

\newcommand{\twb}[1]{\approx_{#1}}
\newcommand{\toc}[1]{\approx^{c}_{#1}}

%% operational rules
 
% Derivation Rules
% 1. parameter is name
% 2. parameter is premise
% 3. parameter is conclusion
% 4. parameter is side conditions
%
\newlength{\lrulename}
\setlength{\lrulename}{0.80cm}
 
\newcommand{\Rule}[4]{\makebox[\lrulename]{{\rm #1}\hfill}
                      $\displaystyle\frac{#2}{#3}\,{#4}$}

% Locality

\newcommand{\loc}[4]{{#1}[ #2 ]^{#3}_{\{#4\}}}
\newcommand{\locv}[4]{{#1}[ #2 ]^{#3}_{#4}}
\newcommand{\nloc}[3]{{#1}[ #2 ]^{#3}}
\newcommand{\lcloc}[3]{{#1}[ #2 ]_{\{#3\}}}
\newcommand{\lncloc}[2]{{#1}[ #2 ]}

%% misc macros

\newcommand{\pair}[2]{\langle {#1},{#2} \rangle}
\newcommand{\threetuple}[3]{\langle {#1},{#2},{#3} \rangle}
\newcommand{\df}{\operatorname{=_{\textrm{df}}}}
\newcommand{\seml}{[\hspace{-0.3ex}[}
\newcommand{\semr}{]\hspace{-0.3ex}]}
\newcommand{\semtwo}[5]{{#2}_{#3}\seml #1 \semr^{#4}_{#5}}
\newcommand{\sem}[4]{{}_{#2}\seml #1 \semr^{#3}_{#4}}

\newcommand{\procin}[2]{on\ #1 \tntin{#2}}
\newcommand{\procout}[2]{on\ #1 \tntout{#2}}
\newcommand{\bin}{\overline{\varovee}}
\newcommand{\bout}{\overline{\varowedge}}
\newcommand{\bopen}{\overline{\varoast}}
\newcommand{\tntin}[1]{\tin\!#1}
\newcommand{\tntout}[1]{\varowedge #1}
\newcommand{\tntopen}[1]{\varoast #1}
\newcommand{\tin}{\varovee}
\newcommand{\tout}{\varowedge}
\newcommand{\topen}{\varoast}
\newcommand{\mobprim}{\text{$\{ \tin, \tout , \topen\}$}}
\newcommand{\pc}{\mid}


%	mpsmacros.tex
%	Extra commands suggested by mps


\newcommand{\ADDTEXT}{\textit{Text inserted.}}
\newcommand{\DELTEXT}{\textit{Text deleted.}}

% \Proofing{initial}{original text}{suggested replacement}
\usepackage{marginnote}

% Inner paragraphs can't include footnotes. Add the footnote text afterwards, e.g.
%     \InnerProofing{MS}{Old text}{New text in braces}
%     \InsertNewText{}
\newcommand{\TheNewText}{}
\newcommand{\InnerProofing}[3]{%
  $^:$#3\stepcounter{footnote}\footnotemark[\value{footnote}]{}%
  \renewcommand{\TheNewText}{#1: #2}%
  \marginnote{\textbf{*\arabic{footnote}*}}%
}
\newcommand{\InsertNewText}{\footnotetext{\TheNewText}}

% Outer paragraphs are much simpler
\newcommand{\OuterProofing}[3]{%
$^:$#3$^:$\stepcounter{footnote}\marginnote{\textbf{*\arabic{footnote}*}}%
\footnotetext[\value{footnote}]{#1:~#2}%
}

% Don't forget to call \InsertNewText{} right after the inner paragraph ends!
\newcommand{\MPSInner}[2]{\InnerProofing{MS}{#1}{#2}}
\newcommand{\AJHInner}[2]{\InnerProofing{AH}{#1}{#2}}

\newcommand{\MPS}[2]{\OuterProofing{MS}{#1}{#2}}
\newcommand{\AJH}[2]{\OuterProofing{AH}{#1}{#2}}



\begin{document}


\section{Equivalence Theory}
\label{sec:equivalence-theory}

What does it mean for two environs $\locv{m}{P}{B}{\vec\rho}$ and
$\locv{n}{Q}{C}{\vec\sigma}$ to be equivalent? We will adopt the standard Morris-style  \cite{Mor68} (or \emph{may-testing} \cite{DNH84}) approach used to characterise
both equivalence of Mobile Ambients \cite{GC99} and barbed congruence in the
$\pi$-calculus \cite{SW01}. Having defined equivalence, we can prove various
safety properties. In particular, we show that $P \Eq Q$ implies that $P$ and
$Q$ have the same type relative to our type system.

\subsection{Immediate Observability}
\label{sec:immediate-observability}

We first define three predicates $\expr \ExhibitsA \alpha$, $\expr \ExhibitsC
\sigma$ and $\expr \ExhibitsE n$, describing the immediate observability of
actions and clocks, and the immediate top-level placement of environs,
respectively. For each predicate $\Exhibits{\bullet}$ we then define a
corresponding contextual equivalence $\Eq_{\bullet}$. Finally, we define
equivalence $\Eq$ for TNT processes to be the conjunction of these three
equivalences. We use the word \emph{symbol} to mean any action, clock or environ
name.

\subsubsection{Actions.}
The notion of immediate observability of an action needs to take account of possible timeouts. For example, the action $a$ can occur immediately in both $a\pref\nil$ and $\timeout{a\pref\nil}{\sigma}{b\pref\nil}$, while the action $b$ can occur immediately in neither. We can capture the required behaviour quite simply, however, by appealing to our semantics.

\begin{definition}
We say that $\expr$ \emph{exhibits} the action $\alpha$, and write $\expr \ExhibitsA \alpha$, provided $\expr \Does{\alpha}$.
\end{definition}


\subsubsection{Clocks.} 

Clocks are used in TNT to delimit segments of behaviour. Consequently, we
consider a clock to be immediately observable if it is possible for it to be the
next clock whose ticking can be seen to occur in the expression's reduction.

\begin{definition}
We say that $\expr$ \emph{exhibits} the clock $\sigma$, and write $\expr \ExhibitsC \sigma$, provided $\expr \Does{\sigma}$.
\end{definition}

\subsubsection{Environs} 

Following Gordon and Cardelli \cite{GC99} we say that $\expr$ \emph{exhibits}
the environ $n$ whenever $n$ is the name of a `top-level' environ. However, our
definition again needs to take account of potential time-outs. For example, $n$
is exhibited in both $\loc{n}{P}{B}{\sigma}$ and
$\timeout{\loc{n}{P}{B}{\sigma}}{\sigma}{\nil}$. In addition, the presence of time-out operators means that restrictions can occur at several locations in the expression, without affecting the environ's visibility.

\begin{definition}
We say that $\expr$ \emph{exhibits} the environ $n$, and write $\expr \ExhibitsE
n$, provided $\expr$ is of one of the forms 
\begin{enumerate}
\item 
    $\locv{n}{\exprb}{B}{\vec{\sigma}}$;
\item
    $\timeout{ \expr' }{\rho}{\exprb}$ where $\expr' \ExhibitsE n$;
\item
    $\stimeout{ \expr' }{\rho}{\exprb}$ where $\expr' \ExhibitsE n$; or
\item
    $\expr' \hide{\vec{\sigma}}$ where $\expr' \ExhibitsE n$.
\item
    $\expr' \res{ \vec{n} }$ where $n \not\in \vec{n}$ and $\expr' \ExhibitsE n$.
\end{enumerate}
\end{definition}


\subsection{Eventual Observability}
\label{sec:eventual-observability}

A symbol $x$ is \emph{eventually observable} (we shall sometimes say it is \emph{revealed}) in an expression $\expr$ precisely when $x$ is the first \emph{relevant} action of the appropriate class.


\subsubsection{Actions.}

There is only one `invisible' action, namely $\tau$. This does not mean, however, that $\tau$ cannot be revealed. Rather, we don't care about \emph{extra} occurrences of $\tau$.

\begin{definition}
We say that $\expr$ \emph{reveals} the action $\alpha$, and write $\expr \RevealsA \alpha$, provided there is some $\expr'$ such that $\expr \DoesMany{\tau} \expr' \ExhibitsA \alpha$.
\end{definition}

\subsubsection{Clocks.}

Clocks are used to delimit activities, so we should allow normal actions to proceed between clock ticks without prejudicing the subsequent tick's right to be considered eventually observed.

\begin{definition}
We say that $\expr$ \emph{reveals} the clock $\sigma$, and write $\expr \RevealsC \alpha$, provided there is some $\expr'$ and some $\vec{\alpha} \in \act^*$ such that $\expr \Does{\vec{\alpha}} \expr' \ExhibitsC \sigma$.
\end{definition}

Since clocks are automatically hidden by enclosure in an environ, and cannot be introduced syntactically except within time-out, the following characterisation is straightforward.

\begin{proposition}
If $\expr \RevealsC$, then $\expr$ is of one of the forms
\begin{enumerate}
\item
    $\timeout{\exprb}{\sigma}{\exprb'}$;
\item
    $\stimeout{\exprb}{\sigma}{\exprb'}$;
\item
    $\timeout{\expr'}{\sigma}{\exprb'}$ where $\expr' \ExhibitsC \sigma$;
\item
    $\stimeout{\expr'}{\sigma}{\exprb'}$ where $\expr' \ExhibitsC \sigma$;
\item
    $\expr' \hide{\vec{\sigma}}$ where $\sigma \not\in \vec{\sigma}$ and $\expr' \ExhibitsC n$.
\item
    $\expr' \res{ \vec{n} }$ where $\expr' \ExhibitsC \sigma$.
\end{enumerate}
\end{proposition}
\begin{proof}
Straightforward. \textbf{Am checking.}
\qed \end{proof}



\subsubsection{Environs.}

An environ is eventually observed provided it can be brought to top level by some sequence of mobility-based reductions.

\begin{definition}
We say that $\expr$ \emph{reveals} the environ $n$, and write $\expr \RevealsE n$, provided there is some $\expr'$ and some $\vec{\ambop} \in \{\tin, \tout, \topen\}^*$ such that $\expr \Does{\vec{\ambop}} \expr' \ExhibitsE n$.
\end{definition}



\subsection{Contextual Congruence}
\label{sec:contextual-congruence}

As usual, a \emph{context} $C$ is an expression with one or more `holes'. We write $[\,]$ for a hole, and $C[\expr]$ for the expression obtained by filling $C$'s holes with copies of $\expr$ (this may result in free names and variables in $\expr$ becoming bound).

\begin{definition}
Let $C$ range over arbitrary contexts. Two expressions are
\begin{enumerate}
\item
    \emph{action equivalent}, written $\expr \EqA \exprb$, provided they reveal the same actions in every context: $C[\expr] \RevealsA \alpha \Leftrightarrow C[\exprb] \RevealsA \alpha$;
\item
    \emph{clock equivalent}, written $\expr \EqC \exprb$, provided they reveal the same clocks in every context: $C[\expr] \RevealsC \sigma \Leftrightarrow C[\exprb] \RevealsC \sigma$;

\item
    \emph{environ equivalent}, written $\expr \EqE \exprb$, provided they reveal the same environs in every context: $C[\expr] \RevealsE n \Leftrightarrow C[\exprb] \RevealsE n$.
\end{enumerate}
\end{definition}

Finally, full equivalence of TNT expressions is defined to be the conjunction of action, clock and mobility equivalence.
\begin{definition}[Expression Equivalence]
Two expressions are (contextually) \emph{equivalent}, written provided they are action-, clock- and environ-equivalent. That is,
    $\expr \Eq \exprb
     \Means
     \left( \expr \EqA \exprb \right)$
     and
     $\left( \expr \EqC \exprb \right)$
     and
     $\left( \expr \EqE \exprb \right)$
\end{definition}

\begin{lemma}
Contextual equiavlence is a congruence.
\end{lemma}
\begin{proof}
Straightforward \textbf{Am checking}
\qed \end{proof}



\bibliographystyle{splncs}
\bibliography{literature}


\end{document}
